\documentclass[a4paper,11pt,twoside]{report}
\usepackage[portuguese]{babel}
\usepackage[utf8]{inputenc}
\usepackage{graphics}
\usepackage {color}
\usepackage{graphicx}
\usepackage{geometry}
\usepackage{flushend}
\usepackage[normalem]{ulem}              % para sublinhar texto
\usepackage{fancyhdr}                    % Para cabeçalhos
\usepackage{url}                         % Para tratar endereços 'url'
\usepackage{kpfonts}
\usepackage{latexsym}
\usepackage{hyperref}                 % For creating hyperlinks in cross references
\usepackage{amssymb}           % carrega letras matemáticas


\begin{document}
\title{A vida do Suricata}
\author{João António-dos-Santos}
\date {\today}
\maketitle

\chapter{introduction}
{

\section {Sobre a Suricata}


{O { \bf{suricata}}, também chamado de   { \bf{suricato}} ou { \bf{suricate}} { \it{(Suricata suricatta}) é um pequeno mamífero da família {\it {Herpestidae}}, nativo do deserto do Kalahari. Estes animais têm cerca de meio metro de comprimento (incluindo a cauda), em média 730 gramas de peso, e pelagem acastanhada. Têm garras afiadas nas patas, que lhes permitem escavar a superfície do chão e dentes afiados para penetrar nas carapaças quitinosas das suas presas. {\textcolor{red} {Outra característica distinta é a sua capacidade de se elevarem nas patas traseiras, utilizando a cauda como terceiro apoio}.



\section{Características gerais}

\begin{itemize}
    \item Alimenta-se principalmente de insetos (cerca de 82 por cento):
    \begin{itemize}
        \item larvas de escaravelhos e de borboletas;
        \item milípedes;
        \item aranhas.
    \end{itemize}
    
    \item ...mas também de:
    \begin{itemize}
        \item escorpiões;
        \item pequenos vertebrados (répteis, anfíbios e aves);
        \item ovos;
        \item matéria vegetal.
    \end{itemize}
    
    \item \underline{São relativamente imunes ao veneno} das najas e dos escorpiões, sendo estes, inclusive, um dos alimentos que mais apreciam.
\end{itemize}

\chapter{Desenvolvimento}
\section{Onde avistar suricatas no habitat selvagem?}
Existem varios parques nacionais em Africa onde e possivel avistar e ate interagir com suricatas no seu habitat selvagem. No entanto, existe uma regra de ouro: os suricatas nao gostam de chuva, por isso prefira dias solarengos.
Em baixo apresenta-se uma lista de parques ordenada por numero de suricatas por Km2:
\begin{enumerate}
\item Parque “Kgalagadi”, Africa do Sul / Botswana
\item Parque nacional “Karoo”, Africa do Sul
\item Reserva do vale magico do Suricata, Africa do Sul 4. Parque nacional Iona, Angola
\end{enumerate}

ou
\begin{description}
\item \textbf{parque1} Parque “Kgalagadi”, Africa do Sul / Botswana 
\item \textbf{parque2} Parque nacional “Karoo”, Africa do Sul
\item \textbf{parque3} Reserva do vale magico do Suricata, Africa do Sul
\item \textbf{parque4} Parque nacional Iona, Angola


\end{description}


\end{document}
