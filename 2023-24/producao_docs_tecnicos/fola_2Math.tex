\documentclass[12pt,a4paper]{article}
\usepackage[utf8]{inputenc}
\usepackage[portuguese]{babel}
\usepackage[T1]{fontenc}
\usepackage{amsmath}
\usepackage{amsfonts}
\usepackage{amssymb}
\usepackage{makeidx}
\usepackage{graphicx}
\usepackage{enumitem} 

\usepackage[left=2cm,right=2cm,top=2cm,bottom=2cm]{geometry}
\author{Omeir Haroon}
\title{Exercicios de PDT \\ ficha 2 -  Modo Matematico}
\begin{document}

\maketitle
\section{simbolos matematicos}
\begin{enumerate}[label = \Alph*.]
\item Se $L(x)$ representar "$x$ tem cabelo louro", entao a frase (1)escreve-se simbolicamente $\forall x, L(x)$. A sua negacao, que poderiamos coloquialmente redigir "nem todas as pessoas tem cabelo louro", escreve-se $\neg \forall x,L(x)$, e é logicamente equivalente a (2), que se escreve $\exists x,\neg L(x)$.

\item Se \textbf{$A \subseteq B$}, entao \textbf{A} e \textbf{B} tem exatamente os mesmos elementos e, portanto, \textbf{$A = B$}

\item Sejam \textbf{$A = 2,3,5,7,9, B = 1,,5,7,9, C = \mathbb{N}, D = {{x \in \mathbb{Z}|x<5 }} $}

\end{enumerate}
\section{equacoes}
\begin{enumerate}[label=\roman*]
\item uma equacao com duas linhas e alinhada pelo sinal de igualdade:

\begin{equation}
\begin{split}
x^2 + z^3 &= \sqrt{2+3y} \\
x \div 5 &= z^{x+2} \pi
\end{split}
\end{equation}

\item Tres equacoes com tres colunas (repare com atencao nos alinhamentos): 


\begin{align}
X_a &= \sqrt{x+y} \qquad X_b &= \pi +3 \qquad X_c &= 2 + X^y \\ 
Z_{ax} &= x^3 + 7 \qquad Z_{ay} &= \sqrt{x^4 + 5Y} + 29 \qquad Y_a &= x-2-y \\
Z_{ax} &= 10 \qquad Z_{ay} &= 35y - 2 \qquad Z_{az} &= 2+y
\end{align}

\end{enumerate}



\end{document}
